Many folks are drawn to tech for the many perks it involves: from all the cool conference swag to high paying salaries and all the new tools and programming languages to learn.  While all of those things are pretty awesome, the one thing that has made me genuinely love and remain in this industry is open source and its community. Thanks to open source and the folks who built it, I found my purpose, my community, and a career I feel proud of, and that has positively impacted many lives. 

Being part of the open-source communities has opened many doors for me. In the same measure, I try to give back to it as much as possible through grassroots advocacy, developer advocacy, mentorship, and governance and leadership work. I am an international keynote speaker, open-source maintainer, chairperson to the Diversity and Scientific Computing Committee, Member of the Pyladies Global Council, Trustee for the UK Python Association, and Python Software Foundation Director. 

But I am much more than a community person; I am also an individual contributor and technical lead at the top of my field (MLOps and Machine Learning Engineering). As a Machine Learning engineer, one of the critical parts of my role is bridging the gaps among data science, data engineering, and DevOps teams and defining consistent processes to improve their interactions and deliver higher value. 
Over the last decade or so, I have worked on highly technical roles, from Machine Learning and Research engineer to advocacy roles and, more recently, Director. Wearing these hats has allowed me to develop extensive technical knowledge, which I have used to support companies and teams to transform their technology and data science initiatives.

As a developer advocate at Microsoft, I was responsible for meaningfully engaging with Data Science and Open Source communities, generating high-quality content, speaking at conferences and fostering communities and grassroots initiatives.  One of my favourite activities was what I used to call "community-driven development", which consisted in advocating on behalf of the users and communities to drive the development of multiple products such as VSCode, Azure Machine Learning and Azure in general.

In my current role as a Director at Quansight Labs, I am responsible for the strategical direction of the company, mentoring and managing a significant team of experienced developers and leading a portfolio of technical projects. In addition to that, I have worked very hard to build a culture of trust, openness, and true inclusion, which has been critical to the growth of a distributed team with a presence in the America, EMEA, and APAC regions. 

I am confident that my previous and current experiences make me a unique and strong candidate for this position. I am an excellent technical individual and keep myself on top of new tools and techniques regularly. I genuinely care for and work towards the health of global Open Source communities. I lead projects to make scientific computing and tech, in general, more equitable and accessible to all by engaging with multiple stakeholders through active outreach work, amplifying the voices of historically marginalised folks,  and also bringing communities and engineers together. 

If I were to join Canonical, I would bring a uniquely big passion for communities. I will drive transformational changes in how the engineering teams work and interact with their users and intersectional communities. I want to extend the already solid accessibility efforts at Canonical and better engage with the disabled community. I would very much appreciate the opportunity to discuss how my unique skills can help Canonical engage with the open-source and broader developer communities in a meaningful and sustainable way.





